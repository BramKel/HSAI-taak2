\documentclass[11pt]{article}

\author{Groep 6:\\
		Niels Desair\\
		Bram Kelchtermans\\
		Dylan Toirkens}
		
\title{\textbf{Studie van meetbare objectieven en ontwerpprincipes van Shneidermann}}

\date{19/10/2016}

\usepackage{graphicx}
\usepackage{parskip}
\usepackage{float}

\begin{document}
	\begin{titlepage}
		
		\newcommand{\HRule}{\rule{\linewidth}{0.5mm}} % Defines a new command for the horizontal lines, change thickness here
		
		\begin{center} % Center everything on the page
			
			\textsc{\LARGE Universiteit Hasselt}\\[1.5cm] % Nme of your university/college
			\textsc{\Large Humane en sociale aspecten van de informatica}\\[0.5cm] % Major heading such as course name
			
			\HRule \\[0.4cm]
			{ \huge \bfseries Studie van meetbare objectieven en ontwerpprincipes van Shneidermann}\\[0.4cm]
			\HRule \\[1.5cm]
			
			\begin{minipage}{0.4\textwidth}
				\begin{flushleft} \large
					\emph{Groep 6:}\\
					Niels \textsc{Desair} \newline
					Bram \textsc{Kelchtermans} \newline
					Dylan \textsc{Toirkens}
				\end{flushleft}
			\end{minipage}
			~
			\begin{minipage}{0.4\textwidth}
				\begin{flushright} \large
					\emph{Datum:}\\
					19 Oktober 2016
					\emph{Academiejaar: } \\
					2016-2017
				\end{flushright}
			\end{minipage}\\[4cm]
			\vspace{40 mm}
			\includegraphics[width=3.0cm]{uhasselt-logo}\\[2.0cm]  
		\end{center}
	\end{titlepage}

\section{Inleiding}
Voor het vak Humane en Sociale Aspecten van de Informatica is ons gevraagd om twee applicaties te vergelijken. Dit op het vlak van de ontwerpprincipes van Norman en op het vlak van metaforen. Deze paper zal bestaan uit drie individuele delen waarna we afsluiten met een gezamelijke conclusie. We hebben besloten om de communicatieapplicaties Skype en Google Hangouts te bespreken.
\newpage

\section{Bespreking Niels}
Individuele bespreking (twee tot drie bladzijden, 800 à 1200 woorden)
\subsection{Ontwerpprincipes van Norman}
\subsection{Metaforen}
\newpage

\section{Bespreking Bram}
Individuele bespreking (twee tot drie bladzijden, 800 à 1200 woorden)
\subsection{Ontwerpprincipes van Norman}
\subsection{Metaforen}
\newpage

\section{Bespreking Dylan}
Individuele bespreking (twee tot drie bladzijden, 800 à 1200 woorden)
\subsection{Ontwerpprincipes van Norman}
\subsection{Metaforen}
\newpage


\section{Conclusie}

\newpage
\end{document}